\documentclass{article}
\usepackage{polski}
\usepackage[T1]{fontenc}
\usepackage[utf8x]{inputenc}
\usepackage{booktabs}
\usepackage{multirow}
\usepackage{graphicx}
\usepackage{textcomp}
\usepackage{eurosym}
\usepackage{float}
\usepackage{adjustbox}
\usepackage{graphics}
\usepackage{tabularx}
\usepackage{rotating}
\usepackage{tabulary}
\usepackage{listings}
\usepackage{amstext}
\usepackage{xcolor}
\usepackage{url,textcomp}
\usepackage{hyperref}
\usepackage{listings}
\usepackage{xcolor}
\lstset { 
language=C++,
backgroundcolor=\color{black!5}, % set backgroundcolor
basicstyle=\footnotesize,% basic font setting
}



\title{Projektowanie algorytmów i metody sztucznej inteligencji - Raport 2}

\author{Damian Ryś}

 
\begin{document}

\maketitle


\tableofcontents
\section{Wstęp}
\hspace{0.5cm}Grupa lab: E12-93c 

Termin zajęć: PN 18:45-20:35

Numer indeksu: 252936

Prowadzący: Dr inż. Piotr Ciskowski
\newpage
\section{Zadanie}
Podstawionym przed nami problemem jest posortowanie danych
uzyskanych z witryny "IMDB" zawierających wyłącznie tytuł oraz ranking.
Bazę będziemy sortować względem rankingu przy pomocy poniższym algorytmów:
\begin{enumerate}
    \item MergeSort
    \item QuickSort
    \item IntroSort
    \item BucketSort
\end{enumerate}
Algorytmy te zostaną omówione dokładniej w dalszej cześci pracy.
Co jest warte odnotowania, baza z której korzystamy zawiera błędne wpisy, których
musimy się pozbyć przed przystąpieniem do sortowania i pomiarów.

\section{Algorytmy}
\subsection{MergeSort}
\subsection{QuickSort}
\subsection{IntroSort}
\subsection{BucketSort}


\section{Złożoności obliczeniowe}

\section{Testy}

\section{Wnioski}

\section{}

\end{document}