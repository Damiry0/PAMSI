\documentclass{article}
\usepackage{polski}
\usepackage[T1]{fontenc}
\usepackage[utf8x]{inputenc}
\usepackage{booktabs}
\usepackage{multirow}
\usepackage{graphicx}
\usepackage{textcomp}
\usepackage{eurosym}
\usepackage{float}
\usepackage{adjustbox}
\usepackage{graphics}
\usepackage{tabularx}
\usepackage{rotating}
\usepackage{tabulary}
\usepackage{listings}
\usepackage{amstext}
\usepackage{xcolor}
\usepackage{url,textcomp}
\usepackage[framed,numbered,autolinebreaks,useliterate]{mcode}



\title{Projektowanie algorytmów i metody sztucznej inteligencji - Raport 1}

\author{Damian Ryś}

 
\begin{document}

\maketitle


\tableofcontents
\section{Wstęp}
Grupa lab: E12-93c zmien

Termin zajęć: PN 18:45-20:35

Numer indeksu: 252936

Prowadzący: Dr inż. Piotr Ciskowski

\section{Zadanie}
Postawionym pzed nami problemem jest wysłanie przez użytkownika A do użytkownika B wiadomości przez Internet.
wiadomość ta powinna zostać wysłana w serii pakietów na komputer użytkownika B.
Problem polega w tym, że pakiety mogą przychodzić z różnych przycznym w losowej kolejności do użytkownika B.
Nasz program powinien zatem:
\begin{enumerate}
    \item Podzielić naszą wiadomość na szereg pakietów
    \item Przelosować pakiety w celu zasymulowania sytuacji
    \item Wczytać do struktury danych nasze pakiety
    \item Przesyłać pakiety w foramcie [klucz,wartość]
    \begin{enumerate}
        \item klucz powininen mieć unikalną wartość
        \item wartość powinna być dowolnego typu (użytkownik powinine mieć możliwość przesłania dowolnej wiadomości)
    \end{enumerate}
    \item Złączyć wiadomość w jedną,prawidłową całość i wyświetlić użytkownikowi
\end{enumerate}
\section{Kolejka Priorytetowa}
Wobec postawionego problemu strukturą ADT, którą wybrałem jest kolejka priorytetowa bazowana na drzewie binarnym.
\subsection{Czym jest kolejka priorytetowa?}


\subsection{Zalety i wady}
Strukturę tą wybrałem ze względu na:
\begin{itemize}
    \item Łatwość implementacji
    \item Złożoność czasową lepszą niż O(n) dla uzyskania min oraz max
    \item Dynamiczną alokację danych
\end{itemize}


\subsection{Implementacja}
\subsection{Złożoności obliczeniowe}
Dla naszego programu zgodnie z załozeniami projektowymi zostały obliczene złożoności czasowe poszczególnych 
funkcji naszej struktury:
//przykład

Tak przedstawia się reszta złożoności oblicziowych naszej struktury:
//tabelka

Nasze wyniki pokrywają się z danymi dostępnymi w internecie // jakiś ref // ,zatem uznaje , że kolejka
została zaimplementowana prawidłowo.


\subsection{Testy}
Za bazę naszych testów posłużyły nam "polskie copy pasty". Teksty zostały podzielone,następnie wymieszane, załadowane do strktury,
by finalnie zostały posortowane i złączone w jedną całość w osobnym pliku wynikowym. Na poniższym przykładzie
został zaprezentowany cykl życia naszego pliku.



Program finalnie przeszedł testy na bazie większej ilości plików i był w stanie prawidłowo odtwrzyć plik, zatem
uznaje, że program działa prawidłowo.

\section{Wnioski}
\section{Bibliografia}
https://bradfieldcs.com/algos/trees/priority-queues-with-binary-heaps/


\end{document}